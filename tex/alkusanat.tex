\makeatletter
\def\input@path{{../}}
\makeatother
\documentclass[../keittiokirja.tex]{subfiles}

\begin{document}

\chapter{Alkusanat}
\label{chp:alkusanat}

Ihan alussahan oli suo, kuokka ja Jussi. Joskin Jussille tuli hetikohta kauhea nälkä 
kaikesta siitä kivisen pellon kuokkimisesta. 

Reseptikirjoissa olevat reseptit on monesti, 
joko helvetin vaikeita, tai niissä käytetään jotain pomeranssin kuorta, jota ei ensinnäkään 
saa mistään kaupasta, ja vaikka saiskin, ni mitä sille ite pomeranssille pitäis tehä? 
Kauheeta haaskausta ostaa juttuja joista käytetään vaan pieni osa.

Tämä on esiversio reseptikirjasta, jota on rukattu poikien kesken ihan vitusti. 
Tähän mennessä reseptejä on säilytetty irtotiedostoina pilvessä ja kaikki muu tieto on kulkenut 
suusta suuhun menetelmällä. Nyt on aika latoa kaikki tämä tietämys yksiin kansiin. 
	
Alkuperäinen idea oli toteuttaa yksinkertaisehkoja 
reseptejä aineksilla, joita löytyy kaupasta helposti. Näin kaikki uusavuttomat opiskelijatkin saisivat 
jotain syödäkseen myös silloin kun opiskelijaravintolat on kiinni. No eihän se tietenkään siinä pysyny, vaan 
ollaan kikkailtu kaiken maailman paskaa mihin tarvihtee erikoiskauppoja--lähinnä aasialaisia 
kauppoja.
    
Koska ruuanlaitossa reseptien rooli on varsin pieni, tämä läpyskä on nyt nimetty keittiökirjaksi, eikä 
reseptikirjaksi. Reseptit unohtuvat helposti, joten ne on asennettu kirjan alkuun (luku~\ref{chp:reseptit}) 
nopeaa faktantarkistusta varten. Myöhemmissä luvuissa käsitellään erinäisiä keittiöön ja ruuanlaittoon 
liittyviä aiheita, jotka kirjoittajat kokevat tärkeiksi, ja jotka ovat lopulta paljon tärkeämpiä mitä 
reseptit. 

Ja niinkuin aina lukiessasi mitä tahansa, ota huomioon että esitetyt mielipiteet ovat kirjoittajien 
omia näkemyksiä aiheesta, eivätkä välttämättä edusta maailmankaikkeuden perimmäistä totuutta.
    
Jokatapauksessa koittakkee pärjätä. Ruoka on hyvettä!

\end{document}