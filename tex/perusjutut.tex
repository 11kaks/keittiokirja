\makeatletter
\def\input@path{{../}}
\makeatother
\documentclass[../keittiokirja.tex]{subfiles}

\begin{document}


\chapter{Perusjutut}
\label{chp:perusjutut}

Kaikkea perusjuttua mitä ei viitti joka resptiin kirjottaa erikseen.

\section{Riisi}

On olemassa muunlaisiakin riisejä kuin Uncle Bens'n pitkäjyvänen kuorittu valkonen riisi. 
Riisi on varsinkin Aasiassa kovasti suosittu hiilihydraatin lähde, joten luotamme vahvasti 
aasialaisten tietotaitoon riisin keittämisessä.

Aasialaiset keittävät riisinsä lähes poikkeuksetta riisinkeittimellä. Suomessa harvalla moisia 
vekottimia on, joten tässä hyvin yksinkertainen ja hyvä tapa keittää riisiä kattilassa:

\begin{itemize}
	\item Mittaa siivilään sopiva määrä riisiä esim. yksi kahvimukillinen.
	\item Mittaa kattilaan tuplamäärä vettä eli esimerkissämme kaksi kahvimukillista.
	\item Laita kattila levylle ja levy päälle.
	\item Huuhdo siivilässä olevaa riisiä juoksevan veden alla kunnes siitä ei enää irtoa 
	kauheasti vaaleaa mäskiä. Tämä mäski on tärkkelystä, joka aiheuttaa riisinkeitossa perinteisen 
	ylikiehumisen. Huuhdottu riisi myös menee tahmeaksi paremmin, jolloin sitä on helpompi syödä 
	puikoilla.
	\item Kippaa huuhdottu riisi kattilaan ja laita levy pienelle (esim. 2/6). Kattilaan voi 
	halutessaan laittaa myös kannen päälle. 
	\item Odota kunnes kaikki vesi on haihtunut, jolloin riisi on valmista.
	\item HUOM! Keittyvään riisiin ei ennen tarjoilua kosketa kauhalla tai sörkitä muutenkaan,
	koska tavoitteena on paakkuuntuva riisi. Riisin keittämisessä ei myöskään käytetä suolaa.
\end{itemize}

Tämä keitto-ohje toimii niin jasmiinille, basmatille, kuin melkein mille muullekin riisille. 
Riisinkeittimessä homma toimii aikalailla samalla tavalla. Noudata keittimen ohjeita.

Jos haluat perinteistä suomalaista irtoriisiä, niin keitä runsaassa vedessä välillä sekoitellen.

\subsection{Sushiriisi}

Sushiriisiä myydään nykyään hyvin varustelluissa ruokakaupoissa. Ja kyllä, sushin valmistamiseen 
täytyy käyttää nimenomaan sushiriisiä. 

\com{Pittää lisätä keittoprosessin kuvaus. Toisaalta sushiriisin keitto-ohjeita kyllä löytyy 
muualtakin.}

\section{Nuudelit}

Jos käsityksesi nuudeleista on semmoinen annoskokoinen pussi pikanuudeleita, olet kovin erehtynyt. 
Vaan tämä luku näyttäkööön sinulle valon. 

Nuudelit toimivat lisäkkeenä melko monessa aasialaisessa ruokahärvelissä, ja joissakin ne ovat 
oikeastaan pääosassa. Nuudeleita löytyy monen muotoisia ja monesta asiasta tehtyjä. Kaikille lienee
oma käyttötarkoituksensa, mutta toimivat kyllä melko monesti ristiin.

Kaikille tuttu pikanuudeli on tehty vehnästä ja muistuttaa jossain määrin japanilaista ramen-nuudelia. 
Ramen ja vastaavat toimivat sekä keitoissa että wokeissa. Jotain tällaisia löydät todennäköisesti 
jopa lähikaupastasi.

Riisinuudeleita löytyy melkeimpä kaikista tavallisista marketeista ja esim. SantaMarian valmistamat 
ovat oikein hyviä. Riisinuudeleita voi käyttää melkeimpä mihin vain nuudeliruokaan, jossa ei ole erityisiä 
nuudeleita määritetty. Ovat huomattavasti maukkaampia ja jännempiä mitä perus vehnäpikanuudelit. 
Keittämisen alkuvaiheessa kannattaa vähän pyöräyttää nudskuja, nää klimppiytyy aika helposti. Samasta 
syystä vesi kannattaa pitää koko keittämisen ajan reilusti kiehuvana.

Udon on niin ikään japanilaisen keittiön tuotteita. Nuudelit on valmistettu vehnästä ja ne ovat 
huomattavasti paksumpia kuin muut tässä esitellyt. Löytyy aasialaisista ruokakaupoista kuivattuna 
ja tuoreena. Toimii sekä keitettynä että paistettuna.

Sobanuudelit on valmistettu joka kokonaan tai osittain tattarijauhosta (eng. "buckwheat"). Perinteisesti
syödään joko keitoissa tai kylmänä dippikastikkeen kera. Sobaa löytyy aasialaisista ruokakaupoista 
ja myös joistain marketeista. Soba on mahottoman hyvettä.

Somen on äärimmäisen ohutta vehnänuudelia. Käyttötarkoitukset samat kuin soballa. 

Sobanuudelit on sitä ja tätä ja tota. Niitä myydään aasiankaupoissa. 
Kannattaa kokeilla resepti~\ref{rcp:soba}. Se on tosi hyve!

\section{Proteiinin lähteet}

\subsection{Lihat}

leikoista jotain yleistä ehkä

\subsection{Jauheliha}

saödlfkj

\subsection{Quorn}

asf

\subsection{Tofu}

asfg


\section{Öljyt}

\subsection{Rypsiöljy}

perus paistoöljy. käytä tätä mieluummin kuin voita koska vältetään eläinperäisiä rasvoja

\subsection{Oliiviöljy}

älä paista. italialaisten ruokien viimeistelyyn ja salaattiin.

\subsection{Seesaminsiemenöljy}

älä paista. viimeistelyyn mausteeksi



\end{document}