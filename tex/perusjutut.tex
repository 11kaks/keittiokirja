\makeatletter
\def\input@path{{../}}
\makeatother
\documentclass[../keittiokirja.tex]{subfiles}

\begin{document}


\chapter{Perusjutut}
\label{chp:perusjutut}

Kaikkea perusjuttua mitä ei viitti joka resptiin kirjottaa erikseen.

\section{Riisi}

On olemassa muunlaisiakin riisejä kuin Uncle Bens'n pitkäjyvänen kuorittu valkonen riisi. 
Riisi on varsinkin Aasiassa kovasti suosittu hiilihydraatin lähde, joten luotamme vahvasti 
aasialaisten tietotaitoon riisin keittämisessä.

Aasialaiset keittävät riisinsä lähes poikkeuksetta riisinkeittimellä. Suomessa harvalla moisia 
vekottimia on, joten tässä hyvin yksinkertainen ja hyvä tapa keittää riisiä kattilassa:

\begin{itemize}
	\item Mittaa siivilään sopiva määrä riisiä esim. yksi kahvimukillinen.
	\item Mittaa kattilaan tuplamäärä vettä eli esimerkissämme kaksi kahvimukillista.
	\item Laita kattila levylle ja levy päälle.
	\item Huuhdo siivilässä olevaa riisiä juoksevan veden alla kunnes siitä ei enää irtoa 
	kauheasti vaaleaa mäskiä. Tämä mäski on tärkkelystä, joka aiheuttaa riisinkeitossa perinteisen 
	ylikiehumisen. Huuhdottu riisi myös menee tahmeaksi paremmin, jolloin sitä on helpompi syödä 
	puikoilla.
	\item Kippaa huuhdottu riisi kattilaan ja laita levy pienelle (esim. 2/6). Kattilaan voi 
	halutessaan laittaa myös kannen päälle. 
	\item Odota kunnes kaikki vesi on haihtunut, jolloin riisi on valmista.
	\item HUOM! Keittyvään riisiin ei ennen tarjoilua kosketa kauhalla tai sörkitä muutenkaan,
	koska tavoitteena on paakkuuntuva riisi. Riisin keittämisessä ei myöskään käytetä suolaa.
\end{itemize}

Tämä keitto-ohje toimii niin jasmiinille, basmatille, kuin melkein mille muullekin riisille. 
Riisinkeittimessä homma toimii aikalailla samalla tavalla. Noudata keittimen ohjeita.

Jos haluat perinteistä suomalaista irtoriisiä, niin keitä runsaassa vedessä välillä sekoitellen.

\subsection{Sushiriisi}

Sushiriisiä myydään nykyään hyvin varustelluissa ruokakaupoissa. Ja kyllä, sushin valmistamiseen 
täytyy käyttää nimenomaan sushiriisiä. 

\com{Pittää lisätä keittoprosessin kuvaus. Toisaalta sushiriisin keitto-ohjeita kyllä löytyy 
muualtakin.}

\section{Nuudelit}

Sobanuudelit on sitä ja tätä ja tota. Niitä myydään aasiankaupoissa. 
Kannattaa kokeilla resepti~\ref{rcp:kaalisoppa}. Se on tosi hyve!

Riisinuudeleista kannattaa tehdä hyvää ruokaa. Lämmitä riittävästi vettä kiehuvaksi, nakkaa nudskut 
sekaan, odota kolmisen minuuttia ja valuta vesi pois valitsemallasi tavalla esim. käyttäen lävikköä 
apuna.

\end{document}