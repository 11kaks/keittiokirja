% Magic to get citations work in slave files.
\makeatletter
\def\input@path{{../}}
\makeatother
\documentclass[../keittiokirja.tex]{subfiles}

%
% Kaikki yksittäiset reseptit kerätään tällä filulla yhteen nippuun 
% käyttäen subfile-komentoa. Tämä on myös ainut tiedosto (itse reseptien 
% lisäksi) joka käyttää xcookybooky-pakettia. Kaikkien muiden pitäisi olla 
% aika perus Latexia.
%

\begin{document}

\chapter{Reseptit}
\label{chp:reseptit}

Ruuan laittaminen ei ole varsinaisesti vaikeaa. Kokeileminen ja summamutikassa asioiden pannuun mättäminen 
on parhaimmillaan erittäin hauskaa; pahimmillaankin aikaansaannos on vain syömäkelvotonta mössöä, jonka 
joutuu heittämään biojätteeseen. Joskus on kuitenkin mukava jos on jotain pientä referenssiä mistä 
lähteä liikenteeseen: varsinkin jos ruuan laittamisesta ei ole vielä kovin paljon kokemusta.

% Tämä korjaa vähän viittauksia ja ne menee melkein oikeelle sivulle. Ilmeisesti tuo book-tyyppi hajotaa.
\phantomsection

% setHeadlines muuttaa otsikointia.
\setHeadlines
{% translation
	% Ainekset
    inghead = Ainekset,
    % Valmistusohje
    prephead = Valmistaminen,
    % Vinkkiotsikko
    hinthead = Vinkkuli,
    % Jos resepti jatkuu useammalle sivulle, tämä tulee vikalle sivulle..
    continuationhead = viimone sivu tätä reseptiä,
    % .. ja tämä ekalle sivulle
    continuationfoot = jatkuupi seuravalla sivulla,
    % Kelle ruokaa syötetään esim. henkilöä
    portionvalue = immeistä,
}

% Defaulttileveys ainesosille on vähän liian kapia. Pidä leveyksien summa alle 
% ykkösen, jotta jää vähän tilaa väliin.
\setRecipeLengths{
preparationwidth=0.4\textwidth,
ingredientswidth=0.5\textwidth,
}


\newpage
\subfile{./tex/reseptit/tomaattinenpata}
% ei uutta sivua koska tämä on variaatio edellisestä
\subfile{./tex/reseptit/currypata}


\newpage
\subfile{./tex/reseptit/bataattisosekeitto}


\newpage
\subfile{./tex/reseptit/soba}

\newpage
\subfile{./tex/reseptit/kaalisoppa}

\newpage
\subfile{./tex/reseptit/kungpaokana}

\newpage
\subfile{./tex/reseptit/pizzapohja}

\newpage
\subfile{./tex/reseptit/thaiahven}


\end{document}