
\begin{recipe}
[% 
    preparationtime = {\unit[30]{min}},
    bakingtime={\unit[3,5]{t}},
    portion = {\portion{?}},
    bakingtemperature={\protect\bakingtemperature{
       % fanoven=\unit[230]{\textcelcius},
        topbottomheat=\unit[200]{°C}
       % topheat=\unit[195]{°C},
       % gasstove=Level 2
       }}
    %calory={\unit[3]{kJ}},
    %source = {Somebody you used know}
]
{Currypatapata}
\label{rcp:currypata}
    
    %\graph
    %{% pictures
    %    small=pic/,     % small picture
    %    big=pic/  % big picture
    %}

    \introduction{%
    	Variaatio reseptistä~\nameref{rcp:tomaattinenpata}. Tää on miun 
        lemppariversio, vaik tomaattinenki on hyvää!
    }
    
    
    \ingredients{%
    \unit[750]{g}       & kanaa \\
	\unit[250-500]{g} 	& bataattia \\
    \unit[250-500]{g}   & porkkanaa \\
    \unit[pari]{kynttä}	& valksipulia \\
    \unit[4-5]{dl}      & kookosmaitoa \\
    1-2                 & limeä \\
    1		            & lihaliemikuutio \\
    \unit[2-3]{tl}      & currytahnaa (punainen tai keltainen) \\
    \unit[2]{tl}        & inkivääritahnaa \\
                        & kaffirlimen lehtiä \\
     					& suolaa \\
    	 				& soijaa  \\
                        & chiliä \\
    }
    
    \preparation{%
   		Muuten samalla tavalla kuin edellä mainitussa ohjeessa, mutta tässä 
        tomaattimurska sekä ruokakerma korvataan kookosmaidolla, eli alussa 
        laitetaan puolet kookosmaidosta pataan ja lopussa puolet. Lisäksi lime 
        puristetaan sekaan vasta aivan viimeiseksi, eli kun pata on ollut vikat 
        puoli tuntia uunissa ja on siis valmis (kai sen voi muulloinkin laittaa, 
        mutta näin mie oon tehny).

        HUOM! Mausteet on tässä ne tärkein, eli maku tulee noista limelehdistä, 
        inkivääristä ja currytahnasta!
    }

    %\hint{    	
    %
    %}
    
\end{recipe}
