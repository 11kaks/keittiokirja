
\begin{recipe}
[% 
    preparationtime = {\unit[30]{min}},
   % bakingtime={\unit[30]{min}},
   % bakingtemperature={\protect\bakingtemperature{
       % fanoven=\unit[230]{\textcelcius},
       % topbottomheat=\unit[195]{°C},
       % topheat=\unit[195]{°C},
       % gasstove=Level 2}
    %    },
    %portion = {\portion{4-5}},
    %calory={\unit[3]{kJ}},
    %source = {Somebody you used know}
]
{Pizzapohja-algoritmi}
\label{rcp:pizzapohja}
    
    %\graph
    %{% pictures
    %    small=pic/sobasalaattiPilkkeet,     % small picture
    %    big=pic/sobasalaattiKupissa  % big picture
    %}
    
    
    \ingredients{%
	\unit[2 \unitfrac{3}{4}]{dl}& vettä \\
				& vehnäjauhoja\\
    \unit[25]{g} & hiivaa \\
    \unit[1]{tl} & sokerii \\
    \unit[1]{tl} & suolaa \\
    pari rkl & jotain öljyä \\
    }
    
    \preparation{%
    Avaa raana. Pistä käsi / sormi valuvan veden alle ja säädä veden lämpöä sillee että tuntuu 
    käden lämpöseltä. Jos hiiva on jääkaappikylmää ni saa olla vähän 
    lämpösämpääkin, koska kylmä hiiva jäähdyttää sitä vettä myöhemmin. Kerää valuvaa vettä 
    astiaan noin 2.75 dl (yleensä ohjeissa on 2.5 dl, mut se on vähän naftisti koko pellin pizzaan).
        
	Kaada vesi kulhoon. Lisää sokeri ja suola nyt, kun kädet ei oo vielä kaikes möyhässä. Hiiva 
	perään ja sit vasemmalla kädellä möyhit sen veden sekaan. Kyllä, oletin juuri oikeakätisyytesi.
	
	Ota avatusta jauhopussista kiinni oikealla kädellä ja ala kaatelemaan jauhoja 
	sekaan möyhien samalla vasemmalla kädellä muodostuvaa taikinoidia. Jauhoa lisätään 
	kunnes taikina irtoilee kulhon reunoista ja pysyy kämmenellä kuitenkaan 
	valumatta ihan kauheesti. Tuntuma löytyy kokeilemalla: syö pizzaa pari viikkoa putkeen.
	
	Öljyä ei laiteta vielä! Sen sijaan kulhon päälle laitetaan taikinankohotuskangas, tai mikäli 
	tälläistä ei ole käytettävissä, voi samaan tarkoitukseen käyttää mitä tahansa kankaan tapaista 
	objektia, jossa pienet taikinamöykyt ei ole haitaksi. 
	
	Tee jotain muuta noin 20 minuuttia (ks. vinkki).

	Taikinan pitäisi olla nyt turvonnut vähintään kaksinkertaiseksi. Kaada taikinan päälle 
	valitsemaasi öljyä sillee jonkin verran. Luokkaa pari ruokalusikallista. Öljy tekee taikinasta 
	vähän kimmoisampaa. Tai jotain. Jokatapauksessa öljyllä on kiva läträtä. 

	Öljyisää taikinaa on tarkoitus murjoa tässä vaiheessa. Murjonnan voi suorittaa vaikkapa työntämällä 
	nyrkin (vasemman käden) voimallisesti taikinan läpi kulhon pohjaan, ja kääntämällä taikinaa, joka
	edellleen pysyy kasassa kämmenellä, ja toistamalla nyrkkitemppu. Periaatteessa mitä enemmän murjot, 
	sitä ilmavampaa taikinasta tulee. Käytännössä parin minuutin murjonnan jälkeen lisämurjominen ei juuri 
	vaikuta. Lopuksi jauhota taikinan pinta ympäriinsä, niin että se on helppo siirtää kulhosta uunipellille, 
	joka voi olla leivinpaperoitu, mutta jos tykkäät hinkata peltiä puhtaaksi ni sit ei tarvi.

	Painele taikina pellille sillee suht ovaalin muotoseksi. Lisää jauhoja pinnalle, koska muuten tarttuu käsiin. 
	Levitä taikina kämmenillä painelemalla koko pellin kokoiseksi lätyksi. Lisää jauhoja aina kun tarttuu käsiin, 
	mutta vaan sen verran ku tarvii, muuten tulee ihan vitun jauhosaa pitzsaa.
	
	Pizzapohja on nyt valmis ja voit siirtyä tomaattimössön ja muiden täytteiden lisäämiseen.
	
	Vaikka tämä on vain pizzapohja-algoritmi, niin lienee hyvä sanoa pari sanaa myös täytteistä.
  Vältä tarvetta tunkea pizzaasi kaikkea mitä kaupasta 
	löytyy: 2--4 täytettä on ihan tarpeeks. Ja niitä pariakaan täytettä ei tarvii lyödä 
  kourakaupalla. Jos täytettä on ihan kauheen paljon, niin et saa ikinä kypsäks 
	alla olevaa taikinaa.
  }
    
  \hint{%
       Keskimääräinen animejakso kestää 20 min ilman alku- ja loppulauluja.
  }
    
\end{recipe}
