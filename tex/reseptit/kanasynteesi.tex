
\begin{recipe}
[% 
    preparationtime = {\unit[?]{min}},
    bakingtime={\unit[?]{min}},
    portion = {\portion{?}},
    %calory={\unit[3]{kJ}},
    %source = {Somebody you used know}
]
{Kookoskanakastikesynteesi}
\label{rcp:kanasynteesi}
    
    %\graph
    %{% pictures
    %    small=pic/,     % small picture
    %    big=pic/  % big picture
    %}

    \introduction{%    	
        Käytä syvää pannua, soossia tulee jonki verran.

        Jasmiiniriisin kans oikein bueno soossi.
    }
    
    
    \ingredients{%
	\unit[400]{g}       & maustamatonta kanaa \\
	1      				& paprika (värialue \unit[600-650]{nm})\\
    \unit[1]{prl}       & ananaspaloja \\
    \unit[400]{ml}      & kookosmaitoa \\
    \unit[1]{rkl}		& soijaa \\
    vitusti             & valkosipulia \\
    \unit[1]{tl}        & currytahnaa (punanen / keltanen) \\
    \unit[1]{tl}        & inkiväärimurskatahnajuttua\\
    5-6   				& kaffirlimetin lehtiä \\
    }
    
    \preparation{%
        Pilko paprikat ja paistele niitä lämpölevyllä. Lisää valkosipulia oman 
        maun mukaan ihan vitun paljon. Pistä ananakset sekaan. Jos et tykkää 
        ananaksista laita silti. Nössö. 

        Kun saat tarpeeksesi paprikoiden 
        pyörittelystä lisää kanat. Paista kypsäksi. Tosin on myös aika hardcore 
        syödä kana raakana.

        Kun koko paska on kypsää, pipetoi sekaan kookosmaito. Lisää jo nyt tässä 
        heti limetinlehtiä, että kerkee maku irrota. 

        Iske sekaan soijakastiketta 
        oman suolantarpeen mukaan. Ihmisen suolantarve on kääntäen verrannollinen 
        pelattujen OW-comppimatsien määrään. 

        Lisää 1 teelusikka currytahnaa ja 
        1 teelusikka inkivääritöhnää. Jos oot kova jätkä ja tykkäät mausteisemmasta 
        laitat toki enemmän. 

        Synteesi on sen verran helppo, että se voidaan suorittaa 
        yhtäaikaa riisisynteesin kanssa. Tai jotain. 
    }

    %\hint{
    %}
    
\end{recipe}
