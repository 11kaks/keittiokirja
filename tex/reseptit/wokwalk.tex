
\begin{recipe}
[% 
    preparationtime = {\unit[15]{min}},
    bakingtime={\unit[3]{min}},
    portion = {\portion{1}},
    %calory={\unit[3]{kJ}},
    %source = {Kimmo}
]
{Wok 'n' Walk ripoff}
\label{rcp:wokwalk}
    
    \graph
    {% pictures
        small=pic/wokwalk2,     % small picture
        big=pic/wokwalk1  % big picture
    }
    \introduction{%
    	Tämä resepti mimikoi Wok 'n' Walk -nimisen ravintolaketun
    	nerokasta konseptia, jossa asiakas tilaa wokin valitsemillaan muutamalla 
    	täytteellä ja himoitsemallaan lisukeriisillä tai -nuudelilla. Ravintolassa 
    	valitut täytteet kasataan syöntiastian kokoiseen astiaan, jonka jälkeen 
    	koko paska wokataan. Wok 'n' Walk löytyy 
    	ainakin Latviasta; jos löydät, käy ihmeessä testaamassa!

    	Varsin toimiva
    	ruoka yhden hengen talouteen, jossa perheellisille tarkoitetuilla pakkauskoilla 
    	ostetuilla raaka-aineilla tuotetun ruuan määrä riittää helposti viikoksi.
    	Ei ole kiva syödä samaa ruokaa koko viikkoa, joten tällä reseptillä, ja muutamalla 
    	raaka-aineella saat tehtyä eri safkaa joka päivä: osta muutamaa erilaista vihannesta 
    	ja yhdistele niitä joka päivä eri tavalla.

    	Reseptin raaka-ainelistaukseen ja määriin ei kannata suhtautua millään vakavuudella, 
    	vaan mieluummin ostaa kaupasta juttui mistä tykkää (tai käyttää vihannesten jämät 
    	joita keittiöstä löytyy). Ainoat oleelliset mitat on kastikkeiden ja öljyn määrä.
    	Vaikka tehdäänkin äärettömän maskuliinista harkonladontaruokaa, niin kannattaa pitää 
    	joku raja suolan saannissa: \unit[1]{tl} (\unit[10]{ml}) 20 \% soijakastiketta sisältää 
    	noin \unit[2]{g} suolaa, joten tämän reseptin suolapitoisuus on noin \unit[3]{g}, 
    	mikä tarkoittaa sitä, että kaks annosta tätä paskaa päivässä menee jo yli saantisuositusten.
    	Siitä huolimatta, näen että on parempi tehä vitun hyvää kasvissafkaa, joka on liian 
    	suolasta ja rasvasta, kuin että tehään hyvää liharuokaa, joka on \emph{jokatapauksessa} liian 
    	suolasta ja rasvasta.

    	Nuudeleina suosittelen riisinudskuja, mutta veikkaisin että mikä tahansa käy. Riisillä
    	vastaava systeemi saadaan aikaiseksi käyttämällä ideaa paistetusta riisistä eli 
    	keittele riisi etukäteen (sen kuuluu olla kylmähköä / huoneen lämpöstä paistettaessa) 
    	ja lisäämällä riisi pannulle paistovaiheen lopuksi. 
    	Samaan settiin kannaattaa sekoittaa silloin myös kananmuna (mikä tosin toiminnee myös 
    	nuudeleiden kanssa).
    }
    
    
    \ingredients{%
	2					& kevätsipulin hännät\\
    pieni nyrkillinen	& papuseipäitä\\
    pari lehteä			& kaalta\\
    \unit[\unitfrac{1}{3}]{pkt} 	& kovaa tofua\\
    1 					& chili	\\
    \unit[2]{kynttä} 	& valksipulia \\
    \unit[1]{tl} 		& soijaa (20\% suolaa)\\
    \unit[1]{tl} 		& random soossi (10\% suolaa) \\
    \unit[1-2]{rkl}		& rypsiölyjä \\
    }
    
    \preparation{%

    	Voit laittaa nuudeliveden kiehumaan ja pannun lämpiämään sillä aikaa kun 
    	peset / kuorit / pilkot valitsemasi raaka-aineet (ohuet teflonpannut 
    	menee kieroksi jos niitä lämmittää ihan hulluna, joten jonkin asteista 
    	varovaisuutta voi noudattaa tässä kohtaa ja pilkkoa paskat etukäteen 
    	valmiiks).

   		Ota esiin astia josta aiot syödä ja täytä se leikatuilla 
    	ruoka-aine-entiteeteillä. Paistettaessa entiteettien tilavuus hieman pienenee, joten 
    	lisukkeet mahtuu vielä kivasti samaan kuppiin. Pilko viimesenä valkosipuli ja 
    	chili, koska ne menee pannuun ekana (ne voi jättää leikkuulaudalle oottelemaan).

    	%irrallinen
    	%Tärkeintä on kuitenkin riittävä määrä öljyä, 
    	%valkosipulia, chiliä ja kastikkeita.

    	Lue loppuresepti etukäteen pariin kertaan, koska seuraavat vaiheet ovat melkoisen 
    	intensiivisiä.

    	Kun pannu on niin kuuma että siihen kaadettu öljy haluaa hypätä samaan sänkyyn sun kanssa, kippaa 
    	chilit ja valkosipulit sisään. Noin \unit[10-20]{s} päästä voit kipata kaiken muun sekaan 
    	(myös tofut voi painaa tässä vaiheessa sisään ilman suurempia traumoja).

    	Lähestulkoon samalla kädenliikkeellä (tai aavistuksen myöhemmin) voi nakata nuudelit nyt 
    	jo kiehuvaan veteen.

    	Jos pannun wokkausta imitoiva heiluttelu ei ole tuttua, niin voit sekoitella sisältöä syömäpuikoilla 
    	tai hätätapauksessa puisella paistinlastalla. Pannun sisältöä on tarkoitus pyörittää jatkuvasti, 
    	mutta välillä voi pariksikymmeneksi sekunniksi unohtaa pyörittää, ja näin toimien 
    	tuottaa herkullista aavistuksen palanutta pintaa. 

    	Nuudelit on valmiita noin \unit[2-3]{min} keittelyllä, joten vähän ennenku otat nuudelit pois 
    	liedeltä dehydraatiotarkoituksiin, niin lisää soija ja randomsoossi wokkauspannulle. 

    	Valuta nuudelit ja yhdistä wokkiin pannulla tai syömäastiassa. Tämä ei yleensä onnistu 
    	kovin helposti, koska nuudelit haluaa olla yhessä möykyssö, mutta yritä siitä 
    	huolimatta: nuudeleihin pitää saada vähän makua. 
    }

    \hint{
    	Kokeile esim. mitä tahansa kaalia (pari kolme lehteä), paprikaa, kevätsipulin tilalle 
    	purjosipulia (normi sipuli vähän liian vahvaa). Tofun voi korvata quornilla, nyhtiksellä,
    	lihalla, kanalla... oikeestaan millä vaan. 
    }
    
\end{recipe}
