
\begin{recipe}
[% 
    preparationtime = {\unit[?]{min}},
    bakingtime={\unit[?]{min}},
    portion = {\portion{4}},
    %calory={\unit[3]{kJ}},
    %source = {Somebody you used know}
]
{Kanaa Hoi sin -kastikkeessa}
\label{rcp:kanaahoisin}
    
    %\graph
    %{% pictures
    %    small=pic/,     % small picture
    %    big=pic/  % big picture
    %}

    \introduction{%    	
        Käytä syvää pannua, soossia tulee jonki verran.

        Jasmiiniriisin kans oikein bueno soossi.
    }
    
    
    \ingredients{%
	\unit[400]{g}       & maustamatonta kanaa \\
	1      				& punainen paprika \\
    \unit[1]{prl}       & ananaspaloja \\
    \unit[3]{rkl}		& soijaa \\
    vitusti             & valkosipulia \\
    pari	       		& chiliä \\
    \unit[2]{rkl}       & sokeria \\
                        & suolaa \\
                        & pippuria \\
    \unit[150]{g}       & Hoi sin -kastiketta \\
    2   				& kanaliemikuutiota \\
             			& vehnäjauhoa \\
    \unit[7-8]{dl}      & vettä \\
    }
    
    \preparation{%
        Paista kana pannulla ja mausta suolalla ja pippurilla (älä laita paljon 
        suolaa, koska soijakastike). 

        Siirrä kana syrjään ja paista pilkotut paprika, 
        chili ja sipuli pannulla. Jos on HC kokki ja tykkää tiskistä voi paistella 
        kanan ja vihannekset yhtä aikaa eri pannuilla. 

        Kun paprikat sun muut pehmenee lisää ananakset ja paistele sopivaksi.

        Lisää vihannesten sekaan Hoi sin -kastike ja soijakastike. Sekoittele hyvin, 
        ja lisää sokeri ja valkosipuli. Sekoita ja lisää kana. 

        Sekoita kanaliemikuutiot 
        noin 7-8 dl:aan kiehuvaa vettä ja kaada liemi pannuun. Anna kiehua vain ihan 
        alhaisella lämmöllä ja suurusta samalla vehnäjauholla. Laitan itse joku puolesta 
        yhteen ruokalusikkaa kerralla. \emph{Käytä sitä siivilää} nii ei tuu kökköjä.
    }

    %\hint{
    %}
    
\end{recipe}
