\begin{recipe}
[% 
    preparationtime = {?},
    bakingtime={?},
   % bakingtemperature={\protect\bakingtemperature{
       % fanoven=\unit[230]{\textcelcius},
       % topbottomheat=\unit[195]{°C},
       % topheat=\unit[195]{°C},
       % gasstove=Level 2}
    %    },
    portion = {\portion{?}},
    %calory={\unit[3]{kJ}},
    %source = {Somebody you used know}
]
{Kung Pao -kanaa}
\label{rcp:kungpaokana}
    
    \graph
    {% pictures
       % small=pic/sobasalaattiPilkkeet,     % small picture
       % big=pic/sobasalaattiKupissa,  % big picture
       % bigpicturewidth = 0.6\textwidth,
       % smallpicturewidth = 0.3\textwidth 
    }
    
    
    \ingredients{%
	\unit[400]{g} 	& kanafilettä\\
    3 				& paprikaa (eri värisiä)\\
    ?				& kevätsipulia\\
    \unit[100]{g} 	& herkkusieniä\\
    \unit[100]{g} 	& cashew-pähkinöitä\\
    ?				& chiliä\\
    ?				& valkosipulia\\
    ?				& inkivääriä\\
    ?				& maissijauhoja\\
    2 				& munan valkuaista\\
    ?				& hoi sin -kastiketta\\
    ?				& soijakastiketta\\
    \unit[1]{tl}	& suolaa\\
    }
    
    \preparation{%
        Pilko kanat mönteiksi. Sekota suola ja maissijauho kulhoon ja pyörittele kanapalat siinä, ja sen 
        jälkeen valkuaisessa, ja paista öljyssä. Kanojen paistuessa ehtii pilkkoa kasvikset (jos on nopee). 
        Siirrä kanat pois pannulta. 
        
        
		Pannulle öljyä ja perään chilit, inkivääri ja pilkottu valkosipuli. Inkivääri palaa nopeesti, joten 
		heitä paprikat ja kevätsipuli melkein heti perään, ni pannu vähän jäähtyy. Pähkinät ja sienet mukaan 
		settiin jonki ajan päästä. Kun kasvit alkaa olla valmiita ni nakkaa kanat ja hoi sin -soossi mukaan. 
		
		Sekottele 1 tl maissijauhoa pariin desiin vettä ja kaada mukaan. Jos ei näytä tarpeeks mäskiseltä, 
		niin lisää vettä. Tämän lisäyksen jälkeen mäskin pitää kiehahtaa, jotta maissijauho sakeentuu.
		
		Siinäpä se vissiin. Ihan vitun hyvää.
    }
    
%    \suggestion[Headline]
%    {%
%        Tässä on ehdotuksen otsikko
%    }
%    
%    \suggestion{%
%        Tässä on ehdotus itsessään ja sisältää mahdollisesti hieman enemmän tekstiä kuin äsken 
%        nähty ehdotuksen otsikko.
%    }
%    
%    \hint{%
%        Tämä resepti saattaisi toimia jollain muullakin makaroonilla, esim. täysjyväspagetilla.
%    }
    
\end{recipe}
