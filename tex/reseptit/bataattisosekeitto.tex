
\begin{recipe}
[% 
    preparationtime = {\unit[15]{min}},
    bakingtime={\unit[15]{min}},
    portion = {\portion{?}},
    %calory={\unit[3]{kJ}},
    %source = {Somebody you used know}
]
{Bataattisosekeitto}
\label{rcp:bataattisosekeitto}
    
    %\graph
    %{% pictures
    %    small=pic/,     % small picture
    %    big=pic/  % big picture
    %}

    %\introduction{%
    %	
    %}
    
    
    \ingredients{%
	1-2 kpl  			& (n. 700 g) bataattia \\
	1-2 kpl				& sipulia \\
    \unit[3]{kynttä} 	& valksipulia \\
    \unit[2]{dl}		& kuivattuja punaisia linssejä \\
    					& jotain kermaisaa \\
     					& suolaa \\
    	 				& mustapippuria \\
    rypsiölyjä 			& paistamiseen \\
    }
    
    \preparation{%
   		Lämmitä mausteita ja pieneksi sipellettyä valkosipulia kasarissa öljyssä hetken 
   		aikaa, lisää sitten hienonnettu sipuli. Kuori bataatit ja pilko ne suunnilleen 
   		samankokoisiksi kikkareiksi. Tässä vaiheessa sipuli on yleensä palanut. Heitä 
   		sekaan bataatit ja linssit, ja lisää sopivasti vettä. Kuumenna kiehuvaksi ja 
   		keittele kunnes bataatit ovat kypsiä, mutta ainakin kymmenen minuuttia (että 
   		linssit pehmenevät).

		Nosta kattila liedeltä ja surauta soppa sauvasekoittimella silkkiseksi. Myös tehosekoitin 
		mahtaa toimia. Nosta kattila takaisin liedelle ja lisää kerma / juusto / kookoskerma 
		/ lastenkoskija. 
		Lämmitä sekoitellen kunnes mössö on taas tasalaatuista. Tarkista maku ja lisää suolaa. 

		Semmoisenaan soppa on ehkä vähän tylsää, kannattaa kokeilla seuraavilla lisukkeilla: 
		siemenet (auringonkukan, seesamin, kurpitsan… paahdettuna tai raakana), pähkinät, 
		krutongit, raejuusto taikka pannulla paistetut halloumin palaset. 

		Bataatin ja sipulin voi vaihtaa aikalailla mihin vaan kasveihin ja mössö on edelleen 
		oikein jees. Jos tekee ihan perinteisestä perunasta, kannattaa koettaa lisätä rosmariinia.
    }

    \hint{    	
		Muita mausteita: kaneli on aika hauska. Kumina, paprika ja kurkuma toimii aina. Kannattaa 
		kokeilla myös currytahnaa, jos on oikein villi fiilis.
    }
    
\end{recipe}
