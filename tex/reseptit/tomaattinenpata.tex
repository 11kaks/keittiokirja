
\begin{recipe}
[% 
    preparationtime = {\unit[30]{min}},
    bakingtime={\unit[3,5]{t}},
    portion = {\portion{?}},
    bakingtemperature={\protect\bakingtemperature{
       % fanoven=\unit[230]{\textcelcius},
        topbottomheat=\unit[200]{°C}
       % topheat=\unit[195]{°C},
       % gasstove=Level 2
       }}
    %calory={\unit[3]{kJ}},
    %source = {Somebody you used know}
]
{Tomaattinen bataattipata}
\label{rcp:tomaattinenpata}
    
    %\graph
    %{% pictures
    %    small=pic/,     % small picture
    %    big=pic/  % big picture
    %}

    \introduction{%
    	Tosi helppo ja vaivaton valmistaa, mut aikaa kannattaa varata kun 
        tarkotus ois pitää pata uunissa aika pitkään!

        Voi syyä sellasenaan, mutta meillä usein kaverina riisi tai pasta.
        Ja bataattia ei sitten voi korvata perunalla!
    }
    
    
    \ingredients{%
    n. \unit[1]{kg}     & lihaa (kana, possu tms.)\\
	\unit[250-500]{g} 	& bataattia \\
    \unit[250-500]{g}   & porkkanaa \\
	1 iso 				& sipuli \\
    \unit[pari]{kynttä}	& valksipulia \\
    \unit[400]{g}       & chilitomaattimurskaa \\
    \unit[2]{dl}        & ruokakermaa \\
    1		            & lihaliemikuutio \\
                        & oreganoa \\
     					& suolaa \\
    	 				& pippuria \\
                        & chiliä \\
    }
    
    \preparation{%
   		Laita uuni lämpenemään. Kuori ja pilko porkkanat ja bataatit noin 
        peukun pään kokoisiksi paloiksi, pilko tarvittaessa myös liha (Luisen lihan 
        voi nakata pataan sellasenaan: luut irtoaa parin tunnin aikana, ja ne voi sitten 
        keräillä padasta pois.). Hienonna 
        sipuli. Ota esille iso teräskattila (tai joku muu iso uunin kestävä pata).

        Kuullota sipuli ja laita pataan. Paista kevyesti liha (pintaan vähän väriä) 
        ja laita pataan. Voit myös paistaa bataatteja ja porkkanoita pienen hetken
        ennen kuin laitat nekin pataan.

        Kaada pataan tomaattimurska ja pilko sinne myös valkosipuli. Kiehauta
        hieman vettä ja liota liemikuutio veteen. Kaada liemi pataan ja lisää vielä sen 
        verran vettä, että lihat ja juurekset suurinpiirtein peittyvät. Mausta vielä lopuksi 
        ja sekoita kunnolla.

        Paista uunissa noin 3 tuntia \emph{ilman kantta}, noin tunnin välein sekoittaen. Ota lopuksi 
        pata uunista, lisää ruokakerma joukkoon ja paista vielä noin puoli tuntia.
    }

    %\hint{    	
    %
    %}
    
\end{recipe}
