
\begin{recipe}
[% 
    preparationtime = {\unit[30]{min}},
    bakingtime={\unit[30]{min}},
   % bakingtemperature={\protect\bakingtemperature{
       % fanoven=\unit[230]{\textcelcius},
       % topbottomheat=\unit[195]{°C},
       % topheat=\unit[195]{°C},
       % gasstove=Level 2}
    %    },
    portion = {\portion{4-5}},
    %calory={\unit[3]{kJ}},
    %source = {Somebody you used know}
]
{Kaalisoppa}
\label{rcp:kaalisoppa}
    
    %\graph
    %{% pictures
    %    small=pic/sobasalaattiPilkkeet,     % small picture
    %    big=pic/sobasalaattiKupissa  % big picture
    %}
    
    
    \ingredients{%
		2     		& paprikaa \\
		2      		& keskikokoista sipulia \\
		2--20 		& CHILIÄ\\
		500 g  		& kevät- tai valkokaalta \\
		1 prk 		& kikherneitä \\
		1 prk 		& papuja esim. kidney \\
		6 kynttä	& valkosipulia \\
        möntti      & inkivääriä \\
        1 dl        & rypsiöljyä \\
        1 dl        & soijasoossia \\
    }
    
    \preparation{%
        Huuhtele chilit ja leikkaa vinottain ellipsoidin muotoisiksi renkuloiksi, koska näin 
        chili leikataan Aasiassa. Asenna sipulat johonkin pikkukuppiin. Kuori valkosipulinkynnet, 
        lyö niitä veitsen lappeella ja leikkaa noin kuutiomillin paloiksi. Kuori inkivääri 
        veitsellä ja leikkaa saman kokoseksi ku valkosipuli. Laita molemmat samaan kuppiin 
        chilien kanssa. 

        Huuhtele paprikat ja leikkaa reiluiksi paloiksi. Laita isohkoon kippoon. Poista kaalista 
        pari päälimmäistä lehteä. Leikkaa reiluiksi paloiksi. Heitä pois (tai keksi joku innovatiivinen 
        käyttötarkoitus) keskeltä tulevat kovat 
        isot möntit niin, että käyttöön jää vaan lehtimäisiä paloja. Laita samaan kippoon paprikoiden 
        kanssa jos mahtuu, tai ota uus kippo. (Tässä vaiheessa opiskelijan keittiöstä voi alkaa 
        kipot loppumaan, joten käytä syviä lautasia, matalia lautasia, mikrokupua, jos se on 
        puhdas jne…) Kuori ja pilko sipuli reiluiksi paloiksi ja laita omaan kippoonsa.

        Laita paistinpannu levylle, ja levy täysille. Laita litra vettä kattilaan (kattilan ois hyvä 
        olla n. 3 l vetoinen) ja levy täysille. Samalla voit lisätä soijakastikkeen kattilaan myös.

        Kun paistinpannu on kuuma, laita sinne osa rypsiöljystä. Kippaa chiliä, valkosipulia ja 
        inkivääriä sisältävä kulho pannulle. Paista parikymmentä sekuntia ja kippaa pannun sisältö kattilaan.

        Taas öjyä pannnuun ja paprikat kehiin. Paistele sen aikaa, että saat pikkusen väriä ja kippaa 
        kattilaan. Vastaava proseduuri jälleen kaalille ja lopuksi vielä sipuleille. Kaiken öljyn tulisi 
        olla nyt käytetty paistamiseen. 

        Kippaa pavut ja kikherneet lävikköön ja huuhtele kunnes ne ei vaahtoa enää ihan älyttömästi. 
        Kaada pavut kattilaan. 

        Paprika ja kaali on hitaimpia kypsymään, joten kun ne on vähän pehmeitä, niin keitto on valmis. 
        Ehkä n. 10 min papujen lisäämisestä. Älä keitä ihan mössöksi niitä!

        Pavut ja kikherneet voi toki korvata lihalla, tai jollain muulla proteiinipitoisella. Jos 
        käytät lihaa, älä käytä paistamiseen öljyä, koska muuten ruuasta tulee liian rasvasta ja sinusta 
        valtava läskipallo.
        
        }
    
%    \suggestion[Headline]
%    {%
%        Tässä on ehdotuksen otsikko
%    }
%    
%    \suggestion{%
%        Tässä on ehdotus itsessään ja sisältää mahdollisesti hieman enemmän tekstiä kuin äsken 
%        nähty ehdotuksen otsikko.
%    }
%    
%    \hint{%
%       Ja jonkin sortin hintti on tämä teksti.
%   }
    
\end{recipe}
