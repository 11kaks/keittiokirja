\begin{recipe}
[% 
    preparationtime = {\unit[30]{min}},
   % bakingtime={\unit[15]{min}},
   % bakingtemperature={\protect\bakingtemperature{
       % fanoven=\unit[230]{\textcelcius},
       % topbottomheat=\unit[195]{°C},
       % topheat=\unit[195]{°C},
       % gasstove=Level 2}
    %    },
    portion = {\portion{2-3}},
    %calory={\unit[3]{kJ}},
    %source = {Somebody you used know}
]
{Soba -salaatti}
\label{rcp:sobasalaatti}
    
    \graph
    {% pictures
        small=pic/sobasalaattiPilkkeet,     % small picture
        big=pic/sobasalaattiKupissa,  % big picture
        bigpicturewidth = 0.6\textwidth,
        smallpicturewidth = 0.3\textwidth 
    }
    
    \introduction{%
        Jo muinaiset japanialaiset tiesivät, että kuumalla kelillä on kiva syödä kylmää nuudelia. 
        Soba on osittain tattarijauhosta tehty nuudeli, jota perinteisesti vedetään juurikin kylmänä. 
    }
    
    \ingredients{%
		\unit[1]{pötkylä}	    & sobaa	\\
		\unit[15]{cm} 	& kurkkua \\
		2 			    & kevätsipulin hännät \\
		\unit[1]{möntti} & inkivääriä\\
		\unit[2]{rkl} 	& soijasoossia \\
		\unit[1]{rkl}   & seesamiöljyä \\
		\unit[2]{rkl}   & seesaminsiementä \\
		\unit[50-100]{g}	    & cashew- tai pekaanipähkinöitä
    }
    
    \preparation{%
        Annostele sopiva määrä sobaa. Tämä on soban kaupasta löytämisen jälkeen reseptin vaikein 
        temppu. Soba on yleensä paketista tullessaan sidottu pötkylöihin, mutta pötkylän 
        koko vaihtelee villisti. Tässä tapauksessa kahdesta pötkylästä tuli kolme annosta, vaikka 
        tavoitteena oli vain kaksi. Muiden ainesten määrää kannattaa tarpeen mukaan muokata 
        pötkylöiden koon mukaan. Laita soba sivuun ja aseta hellasi lämmittämään keitinvettä.
        Pähkinät ja siemenet maistuvat paremmalta paahdettuna, joten tässä vaiheessa voit myös
        paahtaa pähkinät ja siemenet.
        
        Veden lämmetessä leikkaa kurkku, inkivääri ja sipuli. 
        Kaikista näistä olisi kiva saada semmoisia puikkosyötäviä suikaleita. Referenssinä toimiköön 
        kuva tuossa ylempänä. Laita pilkotut kasvit kulhoon ja lisää mukaan soija, öljy ja etikka. Sekoita nämä 
        sekaisin: kasvit imevät makuja ja pikkelöityvät etikassa.
         
        Kun vesi kiehuu, laita nuudelit sinne. Keitä n. 4 minuuttia tai kunnes ovat kypsiä, riippuu 
        hieman nuudelista. Kaada kypsät nuudelit siivilään ja huuhtele niitä kylmällä vedellä samalla 
        kädellä möyhien, kunnes ovat täysin jäähtyneet. Lisää kylmät nuudelit samaan kulhoon muiden 
        aineiden kanssa ja sekoita sisältö sekaisin. 

		Aseta annoksen verran kulhon sisältöä pienempään kulhoon, ripottele päälle siemeniä ja 
		pähkinöitä ja tarjoile.
    }
    
%    \suggestion[Headline]
%    {%
%        Tässä on ehdotuksen otsikko
%    }
%    
%    \suggestion{%
%        Tässä on ehdotus itsessään ja sisältää mahdollisesti hieman enemmän tekstiä kuin äsken 
%        nähty ehdotuksen otsikko.
%    }
%    
    \hint{%
        Tämä resepti saattaisi toimia jollain muullakin makaroonilla, esim. täysjyväspagetilla.
        Paahdetut pähkinät ja siemenet säilyvät huoneenlämmössä suljetussa astiassa
        noin kuukauden.
    }
    
\end{recipe}
