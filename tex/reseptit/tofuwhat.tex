
\begin{recipe}
[% 
    preparationtime = {\unit[15]{min}},
    bakingtime={\unit[15]{min}},
    portion = {\portion{?}},
    %calory={\unit[3]{kJ}},
    %source = {Somebody you used know}
]
{Itämainen tofu-what}
\label{rcp:tofuwhat}
    
    %\graph
    %{% pictures
    %    small=pic/,     % small picture
    %    big=pic/  % big picture
    %}

    \introduction{%
    	Hoidetaas ensin nää muodollisuudet:

        “Tofu? TOFUWHAT?!” - Runteli

        Kaikki on varmaan joskus erehtyny syömään jossain koulun ruokalassa jonkun tofuruuan. Tofu on niissä semmosia mauttomia, veteliä ja traumatisoivia klimppejä. Vaan tällä pienellä ranneliikkeellä saa tehtyä tofusta paljon hyvempää.
    }
    
    
    \ingredients{%
	1                   & sipuli\\
	1                   & paprika \\
    1                   & chili \\
    \unit[2]{dl}		& herneen palkoja (marjatytöltä)\\
    1 peukalo			& inkivääriä \\
    \unit[1]{kynsi}     & valksipulia \\
    \unit[1]{palikka}   & kylmäsavutofua \\
    \unit[\unitfrac{1}{2}]{dl}    & soijaa \\
    \unit[2]{rkl}       & seesamiöljyä \\
    1                   & sitruunan mehut \\
    \unit[2]{tujausta}  & riisiviinietikkaa \\
    }
    
% Näyttäis siltä että tässä paistetaan seesamiöljyssä. Onkohan se ollut tarkotus?? - Kimmo

    \preparation{%
   		Ensin verestetään tofu, koska kuivemmasta tofusta saa rapsakampaa. Poista 
        tofupalikka kuorestaan ja leikkaa se levyiksi. Aseta levyt talouspaperin 
        päälle, ja laita toinen pala paperia rakentamasi systeemin päälle. Laita 
        tämän päälle vaikka leikkuulauta ja sen päälle vielä joku paino. Anna 
        tofumehun poistua levyistä paperiin ainakin kymmenen minuuttia. 

        Odotellessa on hyvä silputa kaikki kasvit. Inkivääri, chili ja valkosipuli 
        niin pieneksi kuin vaan saat ja loput isommiksi. On hyvä muistaa että puikoilla 
        syödessä kasvaa aina ihmisenä paljon enemmän, kannattaa siis ehkä jättää kasvit 
        aika isoiksi että niistä saa kiinni.

        Laita pannu lämpiämään ja leikkele mehustuneista tofulevyistä sopivan kokoisia 
        pikkupalikoita. Kärtsää palikoita \emph{kuivassa} pannussa sen aikaa että saat suunnilleen 
        joka sivulle niihin väriä. Lisää pannuun vähän öljyä ja käristä vielä hetken aikaa: 
        näin saavutetaan optimaalinen rapeus.

        Nakkaa tofut hetkeksi pannulta sivuun ja lisää siihen öljyä. Siis siihen pannuun, ei 
        tofuun. Viskaa öljyyn ensin chili, inkivääri ja valkosipuli ja käristä niitä 
        raivoisasti kymmenisen sekuntia. Heitä perään muut kasvit ja laita lämpö täysille 
        simuloidaksesi oikeaa wokkausvälineistöä. Koeta saada kasveihin vähän väriä, vaikket 
        siinä varmaan onnistukaan. 

        Tee kasvien käristyessä kastike, eli laita noi märät jutut sekasin jossain kipossa. 
        Valmista samalla nuudelit, eli kaada niiden päälle kiehuvaa vettä, laita kansi päälle 
        ja anna olla kaksi minuuttia.

        Kun olet tyytyväinen kasvien väriin, heitä sekaan tofut ja kastike. Ota pannu pois 
        liedeltä ja woksuliini on valmista. Syö nuudelien kera.
    }

    %\hint{
    %}
    
\end{recipe}
