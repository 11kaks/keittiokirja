\makeatletter
\def\input@path{{../}}
\makeatother
\documentclass[../keittiokirja.tex]{subfiles}

\begin{document}

\chapter{Kastikkeet ja tahnat}
\label{chp:kastikkeet}

Tämän teoksen resepteistä monet mainitsevat erikoisia soosseja ja möhniä, 
joiden pääasiallinen tarkoitus on toimia mausteena. Kuivamausteillekin on 
toki paikkansa, mutta itä-aasialainen keittiö tahtoo suosia juuri näitä märkiä 
juttuja. 

Kun soossien käyttöön tottuu, kiinteä suola jää pitkälti hyllyyn pölyttymään.


\section{Soikastike}
\label{sec:soijakastike}

Soijakastike on kaikkien aasialaisten keittiöiden perusmauste, mutta suunnilleen Intiasta länteen 
päin sitä ei juurikaan käytetä. Soijakastike on helvetin hyvää, ja siinä on paljon suolaa.

Suomessa ehkä yleisin soijakastike on Kikkoman, joka on japanilainen soijakastike ja kuuluu 
vaaleisiin soijakastikkeisiin. Sillä on helposti tunnistettava oma maku, johon monet 
suomalaisetkin ovat tottuneet. Vaikka nimestä voisi muuta kuvitella, on vaalea soijakastike väriltään 
melkolailla mustaa.

On olemassa myös tummaa soijakastiketta, jota voi löytää hyvin varustellun marketin aasian osastolta, 
mutta vähintäänkin aasiankaupoista. Tumma soija on oikeesti aivan pirun mustaa ja värjää kaiken mustaksi, 
joten sitä kannattaa käyttää hieman varoen, jos ei halua tuhota sapuskan visuaalista ilmettä aivan 
täydellisesti.

Aasiankaupoista löytää heittämällä kymmentä erilaista kastiketta, joten ei muuta ku kokeilemaan. 

\com{Jos tulee testattua esim sienisoijaa, nii siitä vois jotain kertoa.}

\section{Kalakastike}
\label{sec:kalakastike}

Kalakastike on yksi Thaimaalaisen keittiön perusmausteista. Se on hieman sokerisen tuoksuinen ja 
todella voimakassuolainen (yli 20 \%) kastike. Lisäämällä tätä mihin tahansa keksimääsi reseptiin 
saat kivan thaikkuviban.

Mikäli kalakastike ei ole aiemmin tuttua tavaraa, saattaa sen tuoksu vaikuttaa hieman... jännittävältä. 
Aromi kuitenkin tasoittuu paljon kun kastike lisätään ruokaan ja se lämpiää siellä. 

\section{Osterikastike}
\label{sec:osterikastike}

Toinen yleisesti thaikkukeittiössä ja ainakin Japanissa käytetty kastike. Ei kovin voimakas 
suola (luokkaa 10 \%), joten tän kanssa voi läträtä ihan huoletta. Pannulliseen safkaa voi yleensä pari 
kolme ruokalusikallista törkätä muiden soossien kaveriksi. Osterikastike on tosi jankkia, joten pulloa voi 
joutua paiskomaan aika voimallisesti, jotta tavaran saa sieltä ulos.

\section{Currytahnat}
\label{sec:currytahnat}

Thaimaalaiseen keittiöön kuuluu oleellisesti curryt, joiden valmistaminen käy helpoiten käyttämällä currytahnaa. 
Currytahnoja on punaista, vihreää ja keltaista. Keltainen tahna lienee harvinaisinta näistä. Vihreä on tulisinta ja 
punainen sitten hiukan miedompaa. Ei nuo valmistahnat mitään kovin tulisia ole muutenkaan, mutta tätä voi käyttää 
osviittana sopivan chilimäärän arviointiin eli vihreesen paljo chilii ja punaseen ei nii paljoo.

Lisäksi on olemassa panang-currytahnaa, joskin sitä saattaa olla hieman vaikeampi löytää. Tulisuudeltaan samaa
luokkaa kuin punainen. 

Thaicurrylla ei sitten ole mitään tekemistä intialaisen curryn kanssa, joka tehdään kuivamausteista. Maku ei myöskään 
muistuta millään tavalla toistaan. 

\section{Harissa}
\label{sec:harissa}

Pohjois-Afrikkalainen jonkin verran tulinen tahna, josta en tiedä muuta ku että shakshukassa pirun hyveä. Sisältää 
anista ja reilusti korianteria.

\section{Hoi sin -kastike}
\label{sec:hoisinkastike}

Kiinalainen maustekastike. Jossain määrin makeaa. Aika hyvää. 



\end{document}