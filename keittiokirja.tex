\documentclass[%
a4paper,
%twoside,
11pt,
%epäkommentoi draft jos haluat kuvat mukaan
draft
]{book}

% keittiötyyli.sty määrittelee kaikennäkösiä juttuja. Koske varoen.
\usepackage{keittiotyyli}

% FIXME Resepteihin viittaavat hyperlinkit ei mee ihan kohilleen vaan edeltävään reseptiin. 

\begin{document}

\title{Tiedepoikien keittiökirja}
\author{Kimmo Riihiaho, Leevi Lind}
\maketitle

% Muutetaan abstarktin nimi joksikin muuksi. Vois heittää koko abstract-komennnon poikkeen ja tehdä oman luvun.
%\renewcommand{\abstractname}{Alkusanastelut}
%\begin{abstract}
%    \noindent 
%	Tämä on esiversio reseptikirjasta, jota on rukattu poikien kesken ihan vitusti. 
%	Tähän mennessä reseptejä on säilytetty irtotiedostoina pilvessä ja kaikki muu tieto on kulkenut 
%	suusta suuhun menetelmällä. Nyt on aika latoa kaikki tämä tietämys yksiin kansiin. 
%	
%	Alkuperäinen idea oli toteuttaa yksinkertaisehkoja 
%    reseptejä aineksilla, joita löytyy kaupasta helposti. Näin kaikki uusavuttomat opiskelijatkin saisivat 
%    jotain syödäkseen myös silloin kun opiskelijaravintolat on kiinni. No eihän se tietenkään siinä pysyny, vaan 
%    ollaan kikkailtu kaiken maailman paskaa mihin tarvihtee erikoiskauppoja--lähinnä aasialaisia 
%    kauppoja.
%    
%    Koska ruuanlaitossa reseptien rooli on varsin pieni, tämä läpyskä on nyt nimetty keittiökirjaksi, eikä 
%    reseptikirjaksi. 
%    
%    Jokatapauksessa koittakkee pärjätä. Ruoka on hyvettä!
%\end{abstract}

\newpage
\tableofcontents

\newpage
\subfile{./tex/reseptit}

\newpage
\subfile{./tex/perusjutut}

\newpage
\subfile{./tex/keittiovalineet}

\newpage
\subfile{./tex/kasvit}

\newpage
\subfile{./tex/kastikkeet}

\end{document} 